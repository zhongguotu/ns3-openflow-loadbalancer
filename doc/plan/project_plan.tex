%%%%%%%%%%%%%%%%%%%%%%%%%%%%%%%%%%%%%%%%%
% Thin Sectioned Essay
% LaTeX Template
% Version 1.0 (3/8/13)
%
% This template has been downloaded from:
% http://www.LaTeXTemplates.com
%
% Original Author:
% Nicolas Diaz (nsdiaz@uc.cl) with extensive modifications by:
% Vel (vel@latextemplates.com)
%
% License:
% CC BY-NC-SA 3.0 (http://creativecommons.org/licenses/by-nc-sa/3.0/)
%
%%%%%%%%%%%%%%%%%%%%%%%%%%%%%%%%%%%%%%%%%

%----------------------------------------------------------------------------------------
%	PACKAGES AND OTHER DOCUMENT CONFIGURATIONS
%----------------------------------------------------------------------------------------

\documentclass[12pt]{article} % Font size (can be 10pt, 11pt or 12pt) and paper size (remove a4paper for US letter paper)

\usepackage[protrusion=true,expansion=true]{microtype} % Better typography

\usepackage{mathpazo} % Use the Palatino font
\usepackage[T1]{fontenc} % Required for accented characters
\linespread{1.05} % Change line spacing here, Palatino benefits from a slight increase by default

\makeatletter
\renewcommand{\@listI}{\itemsep=0pt} % Reduce the space between items in the itemize and enumerate environments and the bibliography

\renewcommand{\maketitle}{ % Customize the title - do not edit title and author name here, see the TITLE block below
\begin{center}
{\LARGE\@title} % Increase the font size of the title
\end{center}

\begin{flushright} % Right align
\vspace{40pt} % Some vertical space between the title and author name

{\large\@author} % Author name
\vspace{10pt}
\\\@date % Date

\vspace{40pt} % Some vertical space between the author block and abstract
\end{flushright}
}

%----------------------------------------------------------------------------------------
%	TITLE
%----------------------------------------------------------------------------------------

\title{\textbf{The Simulation of an OpenFlow Load-Balancer}} % Title

\author{\textsc{Haochen Wu, Bo Wu\\Yixin Zhao, Wei Fang}} % Author

\date{\today} % Date

%----------------------------------------------------------------------------------------

\begin{document}

\maketitle % Print the title section

%----------------------------------------------------------------------------------------
%	ESSAY BODY
%----------------------------------------------------------------------------------------
\section*{Introduction}
With the emergence of Software-Defined Networking(SDN), many problems can be handled by a non-dedicated way, which could significantly reduce the cost.
One good example is using an OpenFlow switch to do the load-balancing. Recently it begins to go wild in the industry.
There are many ways to balance the load using OpenFlow switch and we are trying to compare the performance of different methods.
The ability enabled by ns-3 to set the network topology and corresponding parameters of hosts and links makes it suitable for exploring this problem.
We will use latency, throughput and server's load as indicators of the performance.

The main works to be done in our project include:

\begin{enumerate}
\item Using ns-3 to create proper network topology.
\item Writing the program running on the OpenFlow controller.
\item Running the simulation and compare the performance under different circumstances.
\end{enumerate}

\section*{Schedule}

\begin{tabular}{ll}
\hline
\textbf{Time} & \textbf{Work}\\
\hline
\textbf{Oct.21-Oct.27:} & Learn NS-3 and go through some examples. \\
\textbf{Oct.28-Nov.3:} & Learn OpenFlow and study the corresponding part.\\
\textbf{Nov.4-Nov.10:} & Create network topology and write the controller.\\
\textbf{Nov.11-Nov.17:} & Debug and make improvement.\\
\textbf{Nov.18-Nov.24:} & Run simulation and collect data.\\
\textbf{Nov.25-Dec.1:} & Processing data.\\
\textbf{Dec.2-Dec.8:} & File the report.\\
\hline
\end{tabular}

\end{document}
